At present there are several global ocean general circulation models
formulated on unstructured meshes. They include MPAS-Ocean
(\cite{Ringler2013}), ICON-o (\cite{Korn2013}) and FESOM
(\cite{Wang2014}, \cite{Danilov2017}). They use different types of
spatial discretization in horizontal plane.  The discretizations of
ICON-o and MPAS-Ocean are based on discrete normal velocities at the
edges of primary triangular (ICON-o) or dual hexagonal mesh
(MPAS-Ocean), placing scalars at cells of primary or dual mesh
respectively. FESOM works with full velocity vectors placed on cells
of triangular mesh, and scalars on median dual control volumes related
to mesh vertices. In analogy with the 'Arakawa' quadrilateral grids
(\cite{ArakawaLamb1977}) the discretizations discussed here will be
referred to as the hexagonal C grid (MPAS-Ocean), triangular C grid
(ICON-o, even though this model uses a mimetic discretization
different from the usual C grid) and triangular B-grid (FESOM). The
list of unstructured-mesh ocean models and discretizations could have
been extended to include many coastal models, but we do not consider
them here.

The question about advantages and disadvantages of regular
quadrilateral B and C grids has been discussed repeatedly and is
already a subject of textbooks (e.g. \cite{BeckmannHeidvogel}). As is
well known, the B grids win in the representation of the Coriolis
force, where the C grids introduce averaging, but lose in the
representation of pressure gradients, where they introduce additional
averaging compared to C grids. The recent analysis by
\cite{Barhan2018} adds to the understanding of the properties of the B
and C grids by studying how they reproduce known instabilities in the
Eady problem configuration. The resolution of global ocean circulation
models increases and numerous setups are used to simulate eddy-rich
regimes, and the analysis helps to assess efficient resolution of
these two grids (discretizations). \cite{Barhan2018} show that the C
grid discretization gives substantially more accurate growth rates of
symmetric instability, but generates a spurious baroclinic (for
wavevectors aligned with unperturbed flow) instability in the range of
wavenumbers extending up to the cutoff wavenumber. Suppression of the
latter requires a special combination of the fourth order scalar
advection and biharmonic horizontal viscosity. Thus, while both the
grids are well suited to model the standard baroclinic instability,
the C grid can be a better option at very high resolutions when the
symmetric instability can become a part of dynamics.

The question arises as to whether these distinctions in reproducing
instabilities are preserved for triangular (hexagonal) C and B
grids. On one hand, the C grids still offer better locality of
pressure gradient computations than B grids. There are, however, many
differences compared to the case of quadrilateral grids. For example,
there is no pressure gradient averaging on triangular B grids, special
reconstructions are required for the Coriolis term on the hexagonal C
grid (\cite{Thuburn2009},\cite{Ringler2010}) and a special approach is
needed to express the horizontal divergence which is equivalent to
averaging of primary normal velocities (\cite{Korn2017}) on the
triangular C grid. There is an additional important difference,
however. All staggered unstructured grid discretizations maintain
spurious modes which emerge due to a disbalance between vector and
scalar degrees of freedom (DoF). There are either too many scalar DoF
(on the triangular C grid) or too many velocity DoF (on the hexagonal
C grid and triangular B grid). Can these spurious modes modify how the
instabilities are simulated or lead to spurious instabilities on their
own is a question which remains to be answered.

The presented work is inspired by the analysis carried out by
\cite{Barham2018} and aims to extend this analysis to triangular mesh
discretization of ICON-o and FESOM. We study how well triangular-mesh
discretization can represent linear symmetric and baroclinic
instabilities in the Eady problem configuration. Same as in
\cite{Barham2018}, we use hydrostatic and the Boussinesq
approximation, and consider the buoyancy as a single scalar field. We
also consider the effect of several momentum advection schemes,
including the vector-invariant scheme for the B grid, which is shown
to trigger the Hollingsworth instability. ....

Due to the presence of spurious modes the analysis is more cumbersome,
and we present necessary preliminary information and detailed
expressions for elementary discrete operators in section ....

      


Some comments on \cite{Barham2018}, we may place their discussion in
Discussions:

1. They assume that the basic flow is aligned with the mesh. Their
results will be somewhat different for other directions. To estimate
the changes, one should, in principle, consider the case when the flow
is along the mesh diagonal. For triangular meshes, one needs to
compare the direction of basic flow along the triangle side and at
$\pi/6$ to it or perpendicular to it (which is the same because of
mesh symmetry). The directions along the baroclinic and symmetric axes
are not numerically equivalent on triangular meshes, in contrast to
quadrilateral meshes

2. Symmetric instability: If we look at (25)-(37) along the symmetric
axis ($a_x=1$, $\delta_x=0$) and compare with (38)-(40), we conclude
that the only difference is in the third term on the LHS of (35) and
(38). This term contains additional averaging on B grid, and the
effect is as if $M^2$ is reduced compared to C grid. This additional
averaging comes from $w$ averaging. On B grid, we average four $w$
values to the velocity point, and on C grid, only two. Apart from
that, there is some extra averaging in the expression for the surface
pressure ($\varphi$) (in denominator on B and numerator on C). It
would be interest to see which of them are important for the symmetric
instability. It would be strange if $\varphi$ is involved, because of
its barotropic character.

If the problem is due to (35) and (38), it will remain on triangular
meshes. We average $w$ from two cells to get it at velocity points of
C grid. And we average it from three vertices to triangle center on B
grid. While there is more averaging on B compared to C, it is less
than on a quadrilateral mesh. This reduced averaging can of course be
masked by the fact that on C grid the vertical velocity is computed on
triangles, which makes it more localized on C grid. It would be good
to discuss this in the paper.

The question on the accuracy of B vs C with respect to the symmetric
instability, a point \cite{Barham2018} did not touch. While C grid is
more accurate, most of accuracy gain (compared to B grid) is at grid
scales. The fact that C grid produces instability at 25 km and 12.5 km
does not mean a lot, because the instability range is too close to the
cutoff wavenumber, and all this will be killed by dissipation. Even
for the mesh of 3.125 km the scales on which one can rely start
between $k_y=2\pi/(8h)$ and $k_y=2\pi/(4h)$, where the B grid is
already not that bad. In summary, unless we use a very high-order
advection so that the dissipation is really at grid scales, there is
no essential difference.


3. The Hollingsworth instability: The last term on the LHS of (41)
signals about potential problem: Indeed, we do not recover pure
advection of $\mathbf{v}'$. This term is small, but it will be
comparable to everything else at grid scales. The appearance of the
Hollingsworth instability (which is computational symmetric
instability) should be related to this term.  It could be perhaps of
interest to see whether we generate terms of similar type with vector
invariant momentum advection. I will try to look what is possible.

4. Spurious baroclinic instability: If I look at 35-37 and 38-40 along
the baroclinic axis, I should assume $a_y=1$ and $\delta_y=0$. In this
case the difference between B and C is in the additional averaging in
the last term on the LHS in 37 compared to 40. If this instability
were intrinsic for discrete formulations, we could have guessed that B
grid does not have it because of averaging. But what is the reason? I
does not look as related to computational baroclinic instability
caused by vertical averaging, and the fact that it is sensitive to the
advection in the buoyancy equation is not really revealing.


\section{Discussions} 

Spatial discretizations used in atmospheric and ocean modeling are
commonly discussed in the context of shallow water equations (see
e.g.\cite{Randall1994,LeRoux2007,Yu2020,Lapolli2024}). In real-world
applications, additional factors such as advection of scalars and
momentum and vertical discretization come into play. The quadrilateral
A, B and C grids are similar concerning reconstruction of the
quantities to be advected, since staggering does not interfere with
such reconstructions. However, staggering introduces horizontal
averaging of vertical velocity in the vertical advection of
momentum. There is less averaging on C grids than on B grids, but it
is fully absent on collocated (A) grids. All grids need horizontal
averaging of the horizontal advecting velocity for momentum, but A and
B grids need it also for scalars. In the vertical direction, ocean
circulation models traditionally use the so called Lorenz staggering
(see discussion in \cite{BellWhite2017}) independent of the horizontal
discretization. This vertical staggering is known to lead to spurious
instability of computational kind (\cite{BellWhite2017},
\cite{Menesguen2025}) if the vertical resolution is too fine compared
to the horizontal one. The horizontal discretization of momentum
advection may trigger the Hollingsworth instability for certain
variants of the vector-invariant form (\cite{Hollingsworth1983,
Bell2017, Ducousso2017}). The accuracy and stability of particular
spatial discretizations are therefore defined by intricate interaction
of many factors, and their analysis needs a framework that is less
limiting than linearized shallow water equations.

The ability of particular discretizations to approximate the
theoretical growth rate of baroclinic and symmetric instability in the
Eady configuration depends on all factors mentioned above. As shown by
\cite{Barham2018}, the respective analysis presents a framework for
learning how particular features of discretizations influence their
performance. For instance, \cite{Barham2018} show that on
quadrilateral meshes the C grid is more accurate than the B grid in
the representation of symmetric instability, and is only slightly more
accurate in the representation of the growth rate of baroclinic
instability.  An unexpected finding of \cite{Barham2018} is a spurious
instability along the baroclinic axis on quadrilateral C grids. While
\cite{Barham2018} show that on sufficiently fine meshes this
instability is easily suppressed by moderate biharmonic viscosity and
diffusion if a fourth-order advection of buoyancy is used, much higher
viscosity is needed on coarse meshes because the spurious instability
penetrates into the range of wavenumbers of physical baroclinic
instability.

Several new ocean circulation models (FESOM, ICON-o and MPAS-Ocean)
are formulated on unstructured (triangular or hexagonal) meshes and
use different horizontal spatial discretizations. An obvious
motivation for studying the discrete Eady problem for these
discretizations is the question of their accuracy. Some aspects of the
answer can be foreseen, as C grids on such meshes generally imply less
averaging than B grids, same as on quadrilateral meshes. More
important, however, is the question about the consequences of spurious
numerical modes (see e.g. \cite{LeRoux2007,LeRoux2012,Danilov2013})
maintained by staggered discretizations on triangular or hexagonal
meshes. These modes were studied predominantly in the context of
linearized shallow water equations, where they create spurious wave
branches which are neutral and hence can be damped. A real question is
what happens in unstable 3D flows, especially in the light of possible
spurious baroclinic instability found in \cite{Barham2018}. Our study
reveals many unexpected details which are briefly discussed
below. Before doing this we make several remarks on the quadrilateral
meshes, which we also explored for comparison.
 
\subsection{Quadrilateral A, B, C grids}
First, the analysis of \cite{Barham2018} assumes that the basic flow
is aligned with the mesh. The other limiting case is the flow directed
along the mesh diagonal. We found that this case the accuracy is even
slightly improved, so that the results of \cite{Barham2018}hold for
arbitrary orientation of the flow and the mesh.

Second, the discrete linearized equations written down in
\cite{Barham2018} for B and C grids show that on the symmetric axis
the B-grid equations differ mainly by additional averaging of vertical
velocity in the vertical momentum advection and horizontal velocity in
the advection of mean buoyancy in the buoyancy equation. The pressure
gradient averaging on B grid does not affect perturbations along the
baroclinic or symmetric axes. This alone implies that the traditional
thinking about the B and C grids in terms of the accuracy of pressure
gradient is insufficient, and other factors are responsible for the
less accurate behavior of the B grids. To put this result in a broader
context, we analyzed the quadrilateral A grid. In this case the
averaging of vertical velocity in momentum equation drops out, but
pressure gradient is less accurate on the baroclinic and symmetric
axes. The result (not shown here, but reproducible with the scripts in
the Supplementary Material) is that the A grid is similar to the B
grid as concerns the representation of the growth rates of baroclinic
and symmetric instabilities. The averaging leading to the loss of
accuracy (as compared to C grid) is different on A and B grids, but
leads to similar consequences. This also means that the pressure
gradient averaging which is the main drawback of A grids in the
context of shallow water wave dynamics does not really make A grids
worser than B grids. However, similar to the C grid, the A grid
maintains spurious instability on the baroclinic axis. The spurious
part is separated by a gap from the physical one at $a=12.5$ km or
higher, but merges with the physical one for $a=25$ km. Respectively,
biharmonic friction with $V_0=0.01$ m/s is needed on the coarse mesh
to suppress the spurious branch, but already $V_0=0.003$ m/s is
sufficient for $a=12.5$ km. The very similar behavior is found for the
C grid, where $V_0=0.01$ m/s is needed for $a=25$ km and $V_0=0.006$
m/s is needed for $a=12.5$ km in the case of the fourth order method
for buoyancy advection. The fourth order advection of buoyancy is
helpful on the A grid, but its presence is less critical than for C
grids.

Third, although the C grid is more accurate in representing symmetric
instability, it is not accurate enough at high wavenumbers. Since the
simulated instability range continues to the cutoff wavenumber,
additional dissipation will be needed to suppress the instability on
grid scales, as all discrete operators are inaccurate in this case.

\subsection{Triangular A, B and C grids and hexagonal C grid}
Formulated briefly, the main result of our analysis in this work is
that all staggered discretizations (i.e. B and C grids) maintain
spurious instabilities along baroclinic and symmetric axes. All
need dissipation to suppress these modes. The spurious part is
generally a combination of several branches, and we hypothesize that
they are partly related to the spurious modes maintained by these
discretizations. The reader shall be reminded that these modes are created by
the disagreement between the numbers of discrete velocity and pressure
degrees of freedom, which is due to the geometry of triangular or
hexagonal meshes. In contrast to the quadrilateral meshes, the
symmetric axis is also affected, and there is strong sensitivity of
the spurious part to the direction of the flow with respect to the
mesh. Additionally, spurious instabilities violate Galilean
invariance, being sensitive to the addition of a uniform flow to the
flow centered around the mid-depth. This is an indication that the
spurious instabilities involve spurious modes. A possibility for this
violation is created by the representation of discrete velocities. On
triangular and hexagonal C grids, the three normal velocity
projections over-define a vector field (see,
e.g. \cite{Danilov2010,Gassmann2011}), and it can be shown that in
this case the Fourier symbol of the linearized momentum advection has
two physical and one spurious eigenvalue. For triangular B grids,
there are two spurious eigenvalues due to the geometrical difference
in the orientation of any two adjacent triangles. These spurious
eigenvalues do not describe physical advection. Uncovering how the
spurious modes of these staggered discretizations induce spurious
branches of baroclinic and symmetric instabilities might be worth of
attention, but is beyond the scope of the present work. In each case a
relatively moderate dissipation is generally sufficient to suppress
the instability on the baroclinic axis. The dissipation depends on the
resolution, on the orientation of the flow to the mesh and on the
accuracy and implementation of the advective operators. The biharmonic
diffusion and viscosity with coefficients scaling as $V_0l^3$ is
generally sufficient to suppress spurious modes for relatively
moderate values of $V_0$ ($V_0<0.01$ m/s, except for very coarse
meshes), but even these values may already affect the accuracy of
maximum instability increment or the range of wavenumbers where
physical instability is observed.

On the symmetric axis, the spurious part is often attached to the
physical part, and it is difficult to see where the physical branch
ends and the spurious one starts. The associated growth rates depend
on the details of advection and flow direction. The analysis of
eigenvectors might be helpful to disentangle spurious and physical
parts, but it is delayed to future studies. The actual dissipation
needed to suppress spurious symmetric instabilities remains depends on
this analysis, but in any case, the situation is less promising than
the one on the quadrilateral meshes.

Unexpectedly, among the discretizations explored here the collocated
triangular A grid shows the least problematic performance compared to
its staggered counterparts. It suffers from a spurious instability
along the baroclinic axis, but this instability is disconnected from
the physical branch on sufficiently fine meshes ($a=12.5$ km or finer)
and can be easily suppressed, with $V_0=0.002$ m/s on the coarse mesh
($a=25$ km) and already $V_0=0.0007$ m/s for $a=12.5$ m/s. In
contrast, the spurious instabilities of staggered discretizations
often penetrate in the range where physical instabilities are present.

\subsection{Brief comparison of unstructured mesh discretizations}

The discretizations considered here behave similar to quadrilateral
grids in simulating the baroclinic instability. The A and B grids
under-predict the increment of maximum instability on coarse meshes,
but the difference is already small for $a=12.5$ km. The C grids
slightly over-predict the increment on coarse meshes. The main point
is the level and type of dissipation needed to remove spurious
contributions. For example, the B grid needs a special implementation
of the Laplacian operator to suppress its geometrical numerical modes
seen in decoupling of velocities on adjacent triangles. In some cases
it has been found that spurious instabilities can only be efficiently
suppressed only if biharmonic viscosity is accompanied by biharmonic
diffusion.
   
Neither of staggered (triangular B and C and hexagonal C) grids is
optimal for simulations of symmetric instability, and triangular A
grid presents a safer option in this case. Spurious behavior on the
symmetric axis is commonly associated with the Hollingsworth
instability triggered by the use of different stencils for the
vorticity and kinetic energy gradient terms in the vector invariant
form of momentum advection (\cite{Hollingsworth1983, BellPT2017,
Ducousso2017}). \cite{Barham2018} show that the spurious branch
appears on the C grid if the energy and enstrophy conserving
discretization of momentum advection is used, which demonstrates this
problem. The appearance of spurious branches might therefore be
expected for the C grids, as they use the vector invariant form of
momentum advection. However, they also appear on the B grid for a
purely advective or flux forms of momentum advection, which do not
split the momentum advection into two separate contributions (as the
vector invariant forms do). This fact and the violation of Galilean
invariance indicate that the situation is more complicated here and
affected by the presence of spurious modes, or in other words, by the
redundant representation of discrete velocities.  The increments of
symmetric instability simulated by triangular and hexagonal C grids
are higher than by A and B grids, but, as mentioned above, an
additional study is needed to precisely separate the physical and
spurious behavior on staggered grids.

\subsection{The comparison with quadrilateral meshes}
Triangular and quadrilateral A grids are very similar on the
baroclinic and symmetric axes, and similarity continues even for the
magnitude of viscosity and diffusivity needed to remove spurious
instability branches. The quadrilateral B grid is much more robust
than its triangular counterpart as it does not support spurious
instability branches. Despite the absence of pressure gradient
averaging in the triangular case, this discretization encounters more
issues related to its spurious modes. Similarly, triangular and
hexagonal C grids are less robust than the quadrilateral C grid, but
here the main issue is the behavior on the symmetric axis.
 
\subsection{Some considerations}
As we have seen, the redundant representation of velocities in
staggered discretizations on triangular and hexagonal meshes
contributes to the appearance of spurious instability branches in the
discrete Eady problem. One may hope that the use of WENO-type methods
for momentum and tracer advection will suppress possible oscillations
in solutions and thus will efficiently damp geometrical modes
supported by staggered discretizations, with direct effect on spurious
instability branches.

The triangular A and B grids can work on arbitrary triangular meshes,
in contrast to the C grids which need the so called orthogonal
meshes. The latter contain centers of circumscribed circles within
respective triangles. In reality, the A and B grids are even more
general and can be extended to arbitrary polygonal meshes, or mixed
meshes containing polygons of different type. Such an extension can to
some degree be done for C grids, but is much less general because of
the orthogonality constraint. The extension to mixed unstructured
meshes can be a promising further step for triangular B grids. In
fact, mixed unstructured meshes formed predominantly by quadrilateral
elements, are already used in FESOM-C, the coastal branch of FESOM
(\cite{DanilovAndrosov2015,Androsov2019}), where they generally
warrant a more stable performance than purely triangular meshes
without any loss in flexibility.

\subsection{Eddy permitting meshes}
The reasoning about the relationship between the Rossby radius $L_d$
and the mesh resolution depends on the inclusion of $1/\pi$ in the
definition. \cite{Barham2018} use $L_d=NH/f_0=40$ km which does not
include this factor. We therefore avoid using $L_d$, and take the
scale of the maximum of baroclinic instability in the
consideration. It corresponds to $k^*\approx4\times10^{-5}$ 1/m, which
gives $k^*a$=1 for $a$=25 km. In this case, $\lambda^*/a=2\pi$, while
$\ lambda/a=2$ corresponds to the spectral cutoff. Here $\lambda$
denotes the wavelength. This is a relatively coarse mesh, as the scale
of maximum instability is only about 3 cutoff wavelengths, which is
formally eddy permitting. On the quadrilateral B grid, the growth rate
multiplied with $N/M^2$ at this resolution is about 0.26 and it is
0.28 on the C grid where dissipation with $V_0=0.01$ m/s is needed to
suppress the spurious branch. They should be compared to the
theoretical value of 0.31. For $a=50$ km the growth rate is 0.18 on
the B grid and 0.20 on the C grid with the dissipation increased to
$V_0=0.015$ m/s for the C grid. The C grid stays more accurate despite
the dissipation, but the difference between the B and C grids is not
large. If the same dissipation is applied to the B grid, the result
decreases to 0.24 and 0.15 respectively, but this dissipation is not
needed in the present framework. \cite{Barham2018} hypothesize that
the reduced growth rate might be the reason why eddy permitting meshes
(and B grids in particular) simulate insufficient eddy energy
levels. The growth rate is indeed reduced for $a=25$ km compared to
the finer meshes, but it is still large so that it is difficult to
accept that this reduction plays a fundamental role.

A common motivation for the energy backscatter parameterization is the
need to handle over-dissipation in eddy-permitting models. An
illustration is provided by the dissipation spectra in
\cite{Soufflet2016} showing that the maximum of kinetic energy
dissipation due to biharmonic friction occurs approximately where the
energy peaks. We, however, see that the biharmonic viscosity with
typical $V_0$ can only reduce the growth rate of instabilities by
about 10\% for $a=25$ km, and much less if the resolution is
increased. So the over-dissipation is related not so much to the
direct damping of growing instability as to the excessive damping of
adjacent higher wavenumbers and steepening the energy spectra to the
extent that they roll off faster than $k^{-4}$. This moves the peak of
dissipation diagnosed at quasi-equilibrium to the largest
scales. Ultimately the over-dissipation is due to the effect of
scaling of viscosity $\nu=V_0l^3$, which makes it larger at given
wavenumber on coarser meshes.


\section{Conclusion}         
As a follow-up to \cite{Barham2018} we analyzed the discrete Eady
instability problem on triangular A and B grids, for the mimetic
discretization of ICON-o (abbreviated as triangular C grid) and on the
hexagonal C grid. We show that the staggered discretizations
(triangular B and C grids and hexagonal C grid) maintain spurious
instability branches on both baroclinic and symmetric axes. These
branches are at least partly induced by numerical modes of these
discretizations. This is indicated by the loss of Galilean invariance
for these branches. The collocated triangular A grid has a spurious
branch on the baroclinic axis, but no such branches on the symmetric
axis. The spurious instability branches are sensitive to the alignment
between the mean flow and the mesh and to the discretization of the
advection of momentum and scalars.

We demonstrate that these grids have very similar accuracy in
representing the physical branch of baroclinic instability, with C
grids being slightly more accurate than the A and B grids at the
coarsest mesh considered here, similar to quadrilateral meshes. On the
symmetric axis, all staggered grids are contaminated by spurious
branches which penetrate into the region of physical
instabilities. The behavior is similar to that induced by the
Hollingsworth instability, but is, at least partly, related to the
presence of geometrical modes maintained by discretizations.

Moderate biharmonic dissipation of momentum and buoyancy is found to
be sufficient to suppress the spurious branches on the baroclinic axis
if meshes are fine enough, but the dissipation coefficient with $V_0$
as high as 0.01 m/s is needed when the resolution is
coarsened. However, similar levels of dissipation are also needed on
the quadrilateral A and C grids in this case. The level of dissipation
needed to suppress spurious branches on the symmetric axis requires
additional studies, as spurious branches intersect with physical ones
on triangular and hexagonal staggered grids. We only show that
moderate dissipation is sufficient to suppress spurious instability at
spectral end.

We show that triangular and hexagonal C grids do not reach the
performance of quadrilateral C grid. Similarly, triangular B grid does
not reach the performance of quadrilateral B grid. Unexpectedly
enough, the triangular A grid is least problematic and is not any
worse than other grids considered here in simulating instabilities in
the framework of Eady problem.

The main recommendation of this study is that a careful choice of
dissipation is needed to suppress spurious behavior for the staggered
discretizations considered here, especially if instabilities along the
symmetric axis are of interest.
